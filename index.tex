%  --------------------------------------------------------------------------
%  Bachelorarbeit IT Qualitätsmanagement für Webprojekte
%  Created by Marc Egli on 2013-01-25.
%  --------------------------------------------------------------------------

%  --------------------------------------------------------------------------
%  Latex Document Settings
%  --------------------------------------------------------------------------
\documentclass[
11pt, % Schriftgrösse
a4paper, % A4 Papier
BCOR10mm, % Absoluter Wert der Bindekorrektur, z.B. BCOR1cm
DIV14, % Satzspiegel festlegen siehe
       % http://www.ctex.org/documents/packages/nonstd/koma-script.pdf
footsepline = false, % Trennlinie zwischen Textkörper und Fußzeile
                     % bei normalen Seiten
headsepline, % Trennlinie zwischen Kopfzeile und Textkörper
             % bei normalen Seiten
oneside, % Zweiseitig
openright,
halfparskip, % Europäischer Satz mit Abstand zwischen den Absätzen
abstracton, % inkl. Abstract
listof=totocnumbered, % Abb.- und Tab.verzeichnis im Inhaltsverzeichnis
bibliography=totocnumbered % Lit.zeichnis in Inhaltsverzeichnis aufnehmen
]{scrreprt}

\usepackage[automark]{scrpage2} % Gestaltung von kopf- und Fußzeile
\usepackage[ngerman]{babel}
\usepackage[ngerman]{translator}
\usepackage{tocbasic}
\usepackage[utf8]{inputenc}
\usepackage{lmodern} % Latin Modern
\usepackage[T1]{fontenc}
\usepackage{hyphenat}
\usepackage{ae} % Schöne Schriften für PDF-Dateien
\usepackage{multirow} % tabellenzellen zusammenfassen
\usepackage{float}

% Tradmark
\def\TTra{\textsuperscript{\texttrademark}}

%1.5 Zeilenabstand
\usepackage[onehalfspacing]{setspace}

% Festlegung des Seitenstils (scrpage2)
\pagestyle{scrheadings}
\clearscrheadfoot
\automark[]{chapter}

% \lehead{\sffamily\upshape\headmark}
% \cehead{}
% \rehead{}
% \lefoot[\pagemark]{\upshape \pagemark}
% \cefoot{}
% \refoot{}
% \lohead{}
% \cohead{}
\lohead{\sffamily\upshape\headmark}
\lofoot{}
\cofoot{}
\rofoot[\pagemark]{\scshape \pagemark}

% Surround parts of graphics with box
\usepackage{boxedminipage}

% Package for including code in the document
\usepackage{listings}

% If you want to generate a toc for each chapter (use with book)
\usepackage{minitoc}
\usepackage{longtable}

% Abkürzungsverzeichnis erstellen.
\usepackage[printonlyused]{acronym}

% schöne Tabelle zeichnen
\usepackage{booktabs}
\renewcommand{\arraystretch}{1.4} %Die Zeilenabstände in Tabllen angepasst.

% für variable Breiten
\usepackage{tabularx}

% Durchgestrichener Text
\usepackage[normalem]{ulem} %emphasize weiterhin kursiv

% This is now the recommended way for checking for PDFLaTeX:
\usepackage{ifpdf}

\usepackage{eurosym}

\usepackage{natbib}

\usepackage{paralist}

\usepackage{array,ragged2e}

% glossary
\usepackage[toc, numberedsection, acronym, numberline, nonumberlist]{glossaries}

\usepackage[]{hyperref}
\hypersetup{
  bookmarks=true,         % show bookmarks bar?
  unicode=true,           % non-Latin characters in Acrobat’s bookmarks
  pdftoolbar=true,        % show Acrobat’s toolbar?
  pdfmenubar=true,        % show Acrobat’s menu?
  pdffitwindow=true,      % window fit to page when opened
  pdfstartview={FitH},    % fits the width of the page to the window
  pdftitle={Semesterarbeit},   
  pdfauthor={Marc Egli},
  pdfsubject={IT Qualitätsmanagement für Webprojekte},
  pdfcreator={TeX Live 2013},
  pdfproducer={pdfTeX 3.1415926-2.4-1.40.13},
  pdfnewwindow=true,      % links in new window
  colorlinks=true,       % false: boxed links; true: colored links
  % linkcolor=blue,          % color of internal links
  % citecolor=black,        % color of links to bibliography
  % filecolor=magenta,      % color of file links
  % urlcolor=cyan          % color of external links
  linkcolor=black,          % color of internal links
  citecolor=black,        % color of links to bibliography
  filecolor=black,      % color of file links
  urlcolor=black,          % color of external links
}

\ifpdf
    \usepackage[pdftex]{graphicx}
\else
    \usepackage{graphicx}
\fi
\pagenumbering{Alph}
\makeatletter 
\let\orgdescriptionlabel\descriptionlabel 
\renewcommand*{\descriptionlabel}[1]{% 
  \let\orglabel\label 
  \let\label\@gobble 
  \phantomsection 
  \edef\@currentlabel{#1}% 
  %\edef\@currentlabelname{#1}% 
  \let\label\orglabel 
  \orgdescriptionlabel{#1}% 
} 
\makeatother 

\makeglossaries
% %!TEX root = index.tex
\newglossaryentry{Content Management System}
{
  name={Content Management System},
  description={Content Management System}
}

\newglossaryentry{unittest}
{
  name=Unittest,
  plural=Unittests,
  description={Test einer einzelnen Softwarekomponente}
}

\newglossaryentry{real_user_monitoring}
{
  name={Real User Monitoring},
  description={Ein Messverfahren, welches Kennzahlen auf dem Computer der Benutzer einer Webseite misst}
}



\newacronym{ajax}{AJAX}{Asynchronous JavaScript and XML}
\newacronym{api}{API}{Application Programming Interface}
\newacronym{ci}{CI}{Continuous Integration}
\newacronym{cms}{CMS}{Content Management System}
\newacronym{css}{CSS}{Cascading Style Sheets}
\newacronym{dns}{DNS}{Domain Name System}
\newglossaryentry{rum}{type=\acronymtype, name={RUM}, description={Real User Monitoring}, see=[Glossar:]{real_user_monitoring}}
\newacronym{seo}{SEO}{Search Engine Optimization}
\newacronym{spf}{SPF}{Sender Policy Framework}
\newacronym{ssl}{SSL}{Secure Sockets Layer}
\newacronym{url}{URL}{Uniform resource locator}



%  --------------------------------------------------------------------------
%  Start Document
%  --------------------------------------------------------------------------
\title{IT Qualitätsmanagement für Webprojekte}

\author{Bachelorarbeit in Informatik\\
    \\
    Studierender - Marc Egli\\
	Auftraggeber - Silvan Spross\\
    Projektbetreuer - Beat Seeliger\\
	\\
	ZHAW}

\date{Januar 2013 bis ..}


\begin{document}
  \ifpdf
    \DeclareGraphicsExtensions{.pdf, .jpg, .tif}
  \else
    \DeclareGraphicsExtensions{.eps, .jpg}
  \fi
  
  \maketitle
  \begin{abstract}
    %!TEX root = ../index.tex
Für Agenturen, welche viele Webprojekte umsetzen und anschliessend betreuen, ist die Qualitätssicherung nach dem Go-Live eine Herausforderung. Entwickler werden für die aktuellen Projekte benötigt und so bleiben wenig Ressourcen übrig um sicherzustellen, dass auch ältere Projekte noch richtig funktionieren. Diese Arbeit versucht diesem Ressourcenmangel entgegenzuwirken, indem das Überprüfen von alten Projekten so weit wie möglich automatisiert wird. Dazu werden Services von Drittanbietern wie auch Eigenentwicklungen eingesetzt. Jedoch bleibt ein kleiner Anteil an Überprüfungen übrig, welcher nicht automatisiert werden kann. Diese müssen dann jeweils Hand ausgeführt werden um die Qualität sicherzustellen.
  \end{abstract}

  \pagenumbering{Roman}
  
  \tableofcontents
  
  \chapter{Personalienblatt}
  %!TEX root = ../index.tex
\begin{tabbing}
\hspace*{4cm} \= \kill
Name, Vorname: \> {\bf Egli, Marc} \\
Adresse: \> {\bf Altstetterstrasse 257} \\
PLZ, Wohnort: \> {\bf 8048 Zürich} \\
\\
Geburtsdatum: \> {\bf 13.11.1983} \\
Heimatort: \> {\bf Winterthur} \\
\end{tabbing}
  
  \chapter{Bestätigung}
  %!TEX root = ../index.tex
Hiermit bestätige ich, Marc Egli, dass ich diese Bachelorarbeit mit dem Thema ``IT Qualitätsmanagement für Webprojekte'' gemäss freigegebener Aufgabenstellung mit Freigabe vom 20.2.2012 ohne jede fremde Hilfe im Rahmen der gültigen Reglements selbständig erarbeitet habe.\\
\\
Zürich, den \\
\\\\
Marc Egli

  
  \chapter{Einleitung}
  \label{cha:Einleitung}
  \pagenumbering{arabic}
  %!TEX root = ../index.tex

\section{Nomenklatur}
\label{sec:nomenklatur}

Um die Lesbarkeit zu erhöhen, werden in dieser Arbeit folgende Begriffe für bestimmte Personen oder Rollen verwendet:

{\bf Kunde} steht für einen Kunden von allink GmbH. Damit gemeint ist die Person oder die Personen welche auf Kundenseite für das Webprojekt zuständig sind.

{\bf Benutzer} steht für einen Benutzer der Webseite. Diese Person ist in den meisten Fällen nicht genauer bekannt. Einem Benutzer ist normalerweise nicht bewusst, dass allink und nicht der Kunde die Webseite erstellt hat, für ihn stellt die Webseite die Repräsentation des Kunden im Internet dar.

\section{Ausgangslage}
\label{sec:ausgangslage}

Webprojekte geniessen gewöhnlich während ihrer Entstehungsphase eine hohe Aufmerksamkeit. Der zugrundeliegende Programmcode wird normalerweise mit Unittests geprüft um sicherzustellen, dass zukünftige Änderungen keine Regressionen nach sich ziehen. Webseiten werden zum Zeitpunkt der Abnahme durch den Kunden meist einer gründlichen Prüfung unterzogen, sowohl durch allink, wie auch durch den Kunden. Nach der Abnahme und nachdem das Webprojekt online verfügbar ist, werden während den ersten Wochen die Zugriffsstatistiken und die Suchmaschienen-Platzierungen analysiert. Falls diese Analyse grössere Mängel aufdeckt, werden diese in der Regel behoben.

Nach dieser ersten Phase werden Inhalte einer Webseite meist von Personen ohne technisches Wissen unterhalten. Dieses Fehlen von technischem Wissen und die Tatsache, dass sich das Internet um die Webseite weiterentwickelt ist häufig der Grund dafür, dass mit der Zeit Probleme auftreten. Solche Probleme können Kleinigkeiten sein, wie eine längere Ladezeit der Webseite verursacht durch ein nicht fachgerecht komprimiertes Bild, oder dazu führen, dass eine Webseite nicht mehr verfügbar ist.

\section{Problemstellung}
\label{sec:problemstellung}

Zu Beginn der Phase in der ein Kunde seine Inhalte in eine Webseite abfüllt, findet jeweils eine Schulung statt. Dabei wird denjenigen welche die Webseite mit Inhalten füllen, erklärt wie Inhalte eingepflegt werden müssen. In dieser Phase wird versucht alle Unklarheiten welche den Umgang mit dem \acrshort{cms} betreffen zu klären. Jedoch wird bei vielen Webseiten der Inhalt regelmässig verändert oder aktualisiert. Da die Person welche für den Inhalt der Webseite zuständig ist, diese Aufgabe meist nur nebenbei hat, sind einfache Vorgänge wie das Ersetzen eines Textes nach einiger Zeit nicht mehr selbstverständlich. Dadurch werden Inhalte teils nicht wie ursprünglich vorgesehen eingefüllt. Im optimalen Fall hat dies keinen direkten Einfluss auf das Funktionieren der Webseite und steht einer Benutzung der Webseite nicht im Wege. Es kann aber vorkommen, dass dadurch Teile der Webseite nicht mehr richtig funktionieren. Dies kann sich durch eine falsche Darstellung, verminderter Leistung oder im Extremfall durch eine nicht mehr benutzbare Webseite zeigen.

Idealerweise fällt es dem Kunden auf, dass die Webseite nicht richtig funktioniert und das Problem kann behoben werden. Falls das Problem für den Kunden nicht unmittelbar sichtbar ist, kann es vorkommen, dass die Webseite zwar den Anschein macht, richtig zu funktionieren jedoch wichtige Funktionen gestört werden. Dies kann zum Beispiel bedeuten, dass die Webseite nicht mehr von Suchmaschinen indiziert werden kann oder ein Kontaktformular nicht mehr wie gewünscht eine E-Mail versendet.

\section{Ziel der Arbeit}
\label{sec:ziel_der_arbeit}
Da eine nicht funktionierende Webseite den Kunden schlecht repräsentiert, sind die allink und insbesondere der Kunde daran interessiert, dass die Webseite möglichst ohne Probleme läuft. Falls die Webseite nicht erreichbar ist, oder nur eingeschränkt läuft, sollte dies möglichst schnell bemerkt werden um den Fehler zu beheben. Da kleinere Fehler nicht von Benutzern und auch nicht vom Kunden bemerkt werden, soll die Qualitätssicherung von Webprojekten so umgestellt werden, dass solche Fehler erkannt werden. Dadurch entsteht für den Benutzer ein besseres Benutzererlebnis und die Webseite kann den Kunden optimal repräsentieren.

  
  \appendix  
  \listoffigures
  \listoftables
  \lstlistoflistings
  
  \glsaddall
  \printglossaries
  
  \nocite{*}
  \bibliographystyle{alphadin}
  \bibliography{cite}
\end{document}
