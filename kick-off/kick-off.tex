%
%  Kick-Off Protokoll
%
%  Created by Marc Egli on 2013-03-27.
%  Template by Silvan Spross
%
\documentclass[]{scrreprt}
\usepackage[ngerman]{babel}

% Use utf-8 encoding for foreign characters
\usepackage[utf8]{inputenc}

% Setup for fullpage use
\usepackage{fullpage}

% Running Headers and footers
%\usepackage{fancyhdr}

% Multipart figures
%\usepackage{subfigure}

% More symbols
%\usepackage{amsmath}
%\usepackage{amssymb}
%\usepackage{latexsym}

% Surround parts of graphics with box
\usepackage{boxedminipage}

% Package for including code in the document
\usepackage{listings}

% If you want to generate a toc for each chapter (use with book)
\usepackage{minitoc}

% This is now the recommended way for checking for PDFLaTeX:
\usepackage{ifpdf}
\usepackage{url}

\ifpdf
    \usepackage[pdftex]{graphicx}
\else
    \usepackage{graphicx}
\fi

\title{Kick-Off Protokoll}
    
\author{Studierender - Marc Egli\\
    Projektbetreuer - Beat Seeliger\\
    Auftraggeber - Silvan Spross, allink\\
    \\
    ZHAW - Zürcher Hochschule für Angewandte Wissenschaften}
    
\date{27. März 2013}

\begin{document}

    \ifpdf
        \DeclareGraphicsExtensions{.pdf, .jpg, .tif}
    \else
        \DeclareGraphicsExtensions{.eps, .jpg}
    \fi

    \maketitle

    \pagenumbering{arabic}

    % \tableofcontents

    \chapter{Kick-Off Protokoll}

    \section{Bachelorthesis}
    IT Qualitätsmanagement für Webprojekte

    \section{Fragen Kick-Off Meeting}
    Alle Fragen aus dem Kick-Off Meeting konnten geklärt werden:
    \begin{enumerate}
        \item Steht die Auftraggeberin bzw. der Auftraggeber hinter dieser Bachelorarbeit? \\
            {\bf Ja}, das Unternehmen steht hinter dieser Bachelorarbeit.
        \item Sind die fachliche Kompetenz und die Verfügbarkeit der Betreuungsperson sicher- gestellt? \\
            {\bf Ja}, die Betreuungsperson ist verfügbar und auf diesem Gebiet kompetent.
        \item Sind die Urheberrechte und Publikationsrechte (u. a. auch die Ablage in der Präsenzbibliothek des Studiengangs Informatik) geklärt? \\
            {\bf Ja}, die Arbeit kann vollständig öffentlich zugänglich gemacht
            werden.
        \item Bekommt die Studentin oder der Student die notwendige logistische und beratende Unterstützung durch die Auftraggeberin bzw. den Auftraggeber? \\
            {\bf Ja}
        \item Entsprechen das Thema und die Aufgabenstellungen den Anforderungen an eine Bachelorarbeit? \\
            {\bf Ja}
        \item Ist die Arbeit thematisch klar abgegrenzt und terminlich entkoppelt von den Prozessen (des Unternehmens) der Auftraggeberin bzw. des Auftraggebers? \\
            {\bf Ja}, wo nötig wurden Vorarbeiten im Unternehmen bereits getätigt.
        \item Ist eine Grobplanung vorhanden? Sind die nächsten Schritte klar formuliert (von der Studentin oder dem Studenten)? \\
            {\bf Ja}
        \item Ist die Arbeit technisch und terminlich von der Studentin oder dem Studenten umsetzbar? \\
            {\bf Ja}
    \end{enumerate}
    
    \pagebreak
    \section{Weiteres}
      In der Aufgabenstellung soll noch ein Satz umformuliert werden, um klar zu stellen, dass ein evaluiertes Produkt nicht nur eingesetzt, sondern auch mit dem zuvor erstellten Katalog eingerichtet werden soll.
    
    \section{Nächster Termin}
    \begin{tabular}{l r}
        Review Katalogisierung & 17.4.2013 \\
    \end{tabular}
    
\end{document}
