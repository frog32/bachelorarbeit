%!TEX root = index.tex
\newglossaryentry{content_management_system}
{
  name={Content Management System},
  description={Ein Content Management System ist ein Softwaresystem, das zur Verwaltung und Pflege von Inhalten von Webseiten verwendet wird. Es gibt für fast jede Webprogrammiersprache diverse CMS Lösungen, sowohl solche welche Open Source sind, wie andere welche proprietär sind}
}

\newglossaryentry{unittest}
{
  name=Unittest,
  plural=Unittests,
  description={Unittests auch Modultests genannt, werden in der Softwareentwicklung eingesetzt um funktionale Einzelteile automatisiert zu testen\cite[S. 44]{books/daglib/0007083}}
}

\newglossaryentry{real_user_monitoring}
{
  name={Real User Monitoring},
  description={Ein Messverfahren, welches Kennzahlen auf dem Computer der Benutzer einer Webseite misst}
}
\newglossaryentry{sender_policy_framework}
{
  name={Sender Policy Framework},
  description={Das Sender Policy Framework ist ein Framework welches das Fälschen einer Absenderadresse in E-Mails verhindern soll. Dazu werden Policies im DNS Eintrag hinterlegt, um die Server welche E-Mails von einer Domain versenden dürfen zu kennzeichnen.}
}



\newacronym{ajax}{AJAX}{Asynchronous JavaScript and XML}
\newacronym{api}{API}{Application Programming Interface}
\newacronym{ci}{CI}{Continuous Integration}
\newglossaryentry{cms}{type=\acronymtype, name={CMS}, description={Content Management System}, see=[Glossar:]{content_management_system}}
\newacronym{css}{CSS}{Cascading Style Sheets}
\newacronym{dns}{DNS}{Domain Name System}
\newglossaryentry{rum}{type=\acronymtype, name={RUM}, description={Real User Monitoring}, see=[Glossar:]{real_user_monitoring}}
\newacronym{seo}{SEO}{Search Engine Optimization}
\newglossaryentry{spf}{type=\acronymtype, name={SPF}, description={Sender Policy Framework}, see=[Glossar:]{sender_policy_framework}}
\newacronym{ssl}{SSL}{Secure Sockets Layer}
\newacronym{url}{URL}{Uniform resource locator}

\renewcommand*{\glsgroupskip}{}