%!TEX root = ../index.tex

\section{Manuelle Tests}
\label{sec:manuelle_tests}
Zusätzlich zu der bestehenden Go-Live Checkliste wurde eine Checkliste erstellt, welche bei Webprojekten nach dem Go-Live regelmässig durchgegangen wird. Diese Checkliste ist in Tabelle~\ref{tab:recuring_checklist} ersichtlich.

\begin{table}[h!]
  \centering
  \begin{tabular}{l p{10cm}}
  \toprule
    1. & Funktioniert die Webseite und alle Applikationen?\\
  \hline
    2. & Sind die verwendeten Versionen von Django, feincms etc noch aktuell? (nur neuere Patch Releases keine Versionssprünge)\\
  \hline
    3. & Funktioniert die Seite in den neusten Versionen von Safari, Firefox, Chrome und IE?\\
  \hline
    4. & Sind auf der Startseite und den Landingpages offensichtliche Rechtschreibefehler?\\
  \hline
    5. & Sind die verwendeten Bilder richtig aufbereitet?\\
  \hline
    6. & Wurde das Design offensichtlich verletzt?\\
  \bottomrule
  \end{tabular}
  \caption{Checkliste für wiederkehrende Überprüfungen}
  \label{tab:recuring_checklist}
\end{table}

\section{Continuous Integration}
\label{sec:continuous_integration_proof_of_concept}
Da Travis für Unternehmen zur Zeit noch nicht frei verfügbar ist, wurde Travis angeschrieben und unsere Situation geschildert. Darauf erhielt allink die benötigte Einladung um Travis zu benutzen.
Um Travis in den bestehenden Projekt Workflow zu integrieren, wurde im der Basiscodebase welche für jedes Django Softwareprojekt verwendet wurde ein .travis.yml erstellt, welches die genaue Testkonfiguration und den Ablauf beinhaltet.

\section{Monitoring}
\label{sec:monitoring_proof_of_concept}

\section{Auswirkungen}
\label{sec:auswirkungen}
Mit all den neuen Tools wurde auch die Menge an E-Mails welche jeder Entwickler täglich erhält stetig grösser. So versendet zum Beispiel Status Cake jedes mal ein E-Mail, wenn eine zu überwachende Seite nicht mehr erreichbar ist. Dies führte dazu, dass die meisten Entwickler diese E-Mails mittels Regeln direkt in eigene Ordner verschieben lassen. So fällt es nicht mehr direkt auf, wenn eine Webseite nicht mehr erreichbar ist, da das E-Mail welches darauf hinweisen soll, automatisch aus dem Posteingang entfernt wird und in einen eigenen Ordner bewegt wird welche für solche Benachrichtigungen vorgesehen ist. Dadurch bleibt übrigen E-Mails im Posteingang mehr Aufmerksamkeit, jedoch kann es vorkommen, dass ein schwerwiegender Fehler über einen Tag nicht bemerkt wird.

E-Mails welche von Sentry versendet werden, teilweise ignoriert, da es sicher ein Entwickler gibt welcher gerade an diesem Projekt arbeitet.