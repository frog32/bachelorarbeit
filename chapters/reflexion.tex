%!TEX root = ../index.tex

\section{Fazit}
\label{sec:fazit}
Bis auf ein evaluiertes Tool welches erst für einige Testprojekte eingesetzt wird, wird in der allink alles was in dieser Arbeit vorgeschlagen wird umgesetzt. Alle Mitarbeiter der Informatik haben die Neuerungen ohne grosse Skepsis aufgenommen und haben sich auch während der Arbeit stets für die kommenden Neuerungen interessiert. Die Geschäftsleitung und die Leitung Informatik unterstützen die Bemühungen diese Neuerungen und sehen neben den dadurch entstehenden Kosten auch einen deutlichen Mehrwert, sowohl für die allink wie auch für die Kunden der allink.

Ich hätte anfangs nicht gedacht, dass ich in dieser Arbeit mehrere externe Services einsetzen werde. Während dem Erstellen des Fehlerkatalogs wurde jedoch schnell klar, dass dieser nicht durch eine Eigenentwicklung abgedeckt werden kann. Die Entscheidung nur wenige Teile selber zu entwickeln und auf bestehende Services aufzubauen ist auch rückwirkend betrachtet noch immer die einzig richtige Wahl.

\section{Ausblick}
\label{sec:ausblick}
Einige Neuerungen sind schon seit zwei Monaten im Einsatz und werden in der allink schon als selbstverständlich angesehen. New Relic aber wurde erst mit wenigen Testprojekten ausprobiert und wird erst in den nächsten Monaten zusammen mit einem neuen Produktivserver komplett eingeführt werden. Bereits sind von Mitarbeitern der Informatik mehrere Ideen genannt worden welche Informationen man auf dem Status Monitor anzeigen könnte und wie man ihn noch effizienter gestalten könnte. Diese und weitere kleine Verbesserungen werden in der nächsten Zeit umgesetzt.

Da in der allink versucht wird, stets zeitgemässe Projekte umzusetzen, werden wohl auch die Methoden zur Qualitätssicherung einem Wandel unterstellt sein. Da in der allink das Thema Qualitätssicherung ernst genommen wird und eine stetige Verbesserung sämtlicher Prozesse angestrebt wird, wird auch zukünftig Zeit dafür investiert werden.

\section{Danksagung}
\label{sec:danksagung}
Ich bedanke mich bei allen welche mich während meiner Bachelorarbeit unterstützt haben. Speziell bedanken möchte ich mich bei Matthias Kestenholz und Stefan Reinhard die mich mir ihrer Erfahrung mit Fehlern bei Webprojekten unterstütz haben. Bei Silvan Spross bedanke ich mich für seinen Einsatz um die Geschäftsleitung und allink als ganzes gegenüber dieser Arbeit zu repräsentieren, und dafür dass er mir stets den Rücken freigehalten hat falls ich für diese Arbeit spontan noch Ferientage benötigte oder seinen Rat brauchte. Ich bedanke mich bei der Geschäftsleitung der allink für die Unterstützung welche mir in Form von Arbeitszeit und finanziellen Mitteln gewährt wurde für die Evaluation und den Proof of Concept. Bei dem ganzen Informatikteam und insbesondere bei Jérémie Blaser bedanke ich mich für das Feedback und die Anregungen für den Status Monitor.

Auch bedanken möchte ich Sandro Iovanna der mir als mein Notebook defekt war kurzerhand sein Notebook ausgeliehen hatte bis die benötigten Ersatzteile eingetroffen waren. Er hat mir dadurch ermöglicht auch in diesen zwei Wochen an meiner Arbeit weiterzuarbeiten. Ein weiteres Dankeschön geht an all die Mitstudenten welche mit mir bereichernde Diskussionen zum Thema Qualitätsmanagement bei Webprojekten geführt haben und speziell bei Alain Horner der für das gegenlesen meiner Arbeit.

Selbstverständlich bedanke ich mich bei Beat Seeliger der diese Arbeit betreut hat und sich viel Zeit dafür genommen hat. Ich hoffe dass wir auch nach dem Studium noch in Kontakt bleiben.

Spezieller Dank geht an meine Freundin, dass sie mich immer nach Kräften unterstützt hat und mir in Zeiten erhöhter Belastung Rückhalt gegeben hat. Auch möchte ich ihr und meinem Bruder für das Korrekturlesen danken.
