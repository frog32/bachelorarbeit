%!TEX root = ../index.tex

\section{Kontext}
\label{sec:kontext}


\newcounter{fnumber} \setcounter{fnumber}{0}
\renewcommand\thefnumber{F\arabic{fnumber}}
\newcommand{\newfnumber}[4]%
{%
\midrule%
\refstepcounter{fnumber}\label{f:#2}%
\thefnumber & #1 & #3 & #4 \\
}



Mögliche Kategorien:

\subsection{Typ}
\begin{itemize}
  \item Webapplikation
  \item Webseite
\end{itemize}

\subsection{Impact}
\label{sub:impact}
Der Schaden welcher durch einen Fehler entstehen kann wurde unabhängig der genauen Funktion einer Webseite in drei Kategorien eingeteilt. Dies lässt ausser acht, dass eine nicht verfügbare Seite bei einem Onlineshop einen Ausfall an Umsatz bedeutet und während der genaue Schaden schwierig zu eruieren ist bei einer Webseite welche die Funktion einer Online-Visitenkarte hat.

\begin{itemize}
  \item 3 Site nicht verfügbar
  \item 2 Seite eingeschränkt verfügbar
  \item 1 unwesentlich
\end{itemize}

\subsection{Frequenz}
\label{sub:frequenz}
Die Frequenz der Fehler wurde in vier Kategorien unterteil. Diese sind in Tabelle~\ref{tab:frequency} ersichtlich.


\begin{itemize}
    \item 1 jährlich
    \item 2 monatlich
    \item 3 wöchentlich
    \item 4 täglich
\end{itemize}


\section{Backend}
\label{sec:backend}

\subsection{SSL}
\label{sub:fehler_ssl}

\begin{longtable}{l>{\raggedright}p{8cm} r r}
    \toprule \textbf{Nr.} & \textbf{Bezeichnung} & \textbf{Impact} & \textbf{Frequenz} \\
    \newfnumber{Zertifikat ausgelaufen}{zertifikat_ausgelaufen}{3}{1}
    \newfnumber{Zertifikat ungültig}{zertifikat_ungultig}{3}{1}
    \newfnumber{SSL nicht erzwungen}{ssl_nicht_erzwungen}{2}{1}
    \newfnumber{Externe Assets ohne SSL}{externe_assets_ohne_ssl}{2}{1}
    \bottomrule
    \caption[Mögliche Fehler SSL]{Mögliche Fehler SSL}
    \label{tab:fehler_ssl}
\end{longtable}


\subsubsection{Zertifikat ausgelaufen}
\label{ssub:zertifikat_ausgelaufen}
Ein ausgelaufenes Zertifikat führt auf dem Browser zu einer Fehlermeldung und ist grundsätzlich ein Sicherheitsrisiko. Da ein SSL Zertifikat normalerweise eine Gültigkeit von einem oder mehreren Jahren hat, sind diese Fehler eher selten.

\subsubsection{Zertifikat ungültig}
\label{ssub:zertifikat_ungültig}
Ein Ungültiges Zertifikat kann durch den Wechsel einer Domain entstehen. Weniger oft kann ein ungültiges Zertifikat auch dadurch entstehen, dass das Zertifikat widerrufen wurde. Der Browser zeigt dann normalerweise eine Warnmeldung an.

\subsubsection{SSL nicht erzwungen}
\label{ssub:ssl_nicht_erzwungen}
Ein SSL Zertifikat nützt wenig wenn der Benutzer nicht auf HTTPS umgeleitet wird.

\subsubsection{Externe Assets ohne SSL}
\label{ssub:externe_assets_ohne_ssl}
Bei einer Webseite welche über mit SSL geschützt ist, ist es wichtig, dass jede Ressource welche nachgeladen wird HTTPS unterstützt. Ansonsten zeigen die meisten Browser eine Warnung an, da dies ein Sicherheitsloch darstellt.

\subsection{DNS}
\label{sub:fehler_dns}

\begin{longtable}{l>{\raggedright}p{8cm} r r}
    \toprule \textbf{Nr.} & \textbf{Bezeichnung} & \textbf{Impact} & \textbf{Frequenz} \\
    \newfnumber{Domain ausgelaufen}{domain_ausgelaufen}{3}{1}
    \newfnumber{DNS Server nicht verfügbar}{dns_server_nicht_verfugbar}{3}{1}
    \newfnumber{DNS Eintrag fehlerhaft}{dns_eintrag_fehlerhaft}{3}{1}
    \newfnumber{SPF Eintrag fehlerhaft}{spf_eintrag_fehlerhaft}{2}{1}
    \bottomrule
    \caption[Mögliche Fehler DNS]{Mögliche Fehler DNS}
    \label{tab:fehler_dns}
\end{longtable}

\subsubsection{Domain ausgelaufen}
\label{ssub:domain_ausgelaufen}
Domainnamen werden gewöhnlich im Jahresrythmus bezahlt. Bei nicht bezahlen der Kosten wird einem die Domain wieder entzogen was dazu führt, dass der DNS Eintrag entfernt wird.

\subsubsection{DNS Server nicht verfügbar}
\label{ssub:dns_server_nicht_verfügbar}
Für den Fall, dass der DNS Server nicht verfügbar ist, ist dieser mehrfach vorhanden.

\subsubsection{DNS Eintrag fehlerhaft}
\label{ssub:dns_eintrag_fehlerhaft}

\subsubsection{SPF Eintrag fehlerhaft}
\label{ssub:spf_eintrag_fehlerhaft}

\subsection{Hintergrundprozesse}
\label{sub:fehler_hintergrundprozesse}

\begin{longtable}{l>{\raggedright}p{8cm} r r}
    \toprule \textbf{Nr.} & \textbf{Bezeichnung} & \textbf{Impact} & \textbf{Frequenz} \\
    \newfnumber{Cronjob Fehler}{cronjob_fehler}{1}{2}
    \newfnumber{Worker Fehler}{worker_fehler}{2}{2}
    \bottomrule
    \caption[Mögliche Fehler Hintergrundprozesse]{Mögliche Fehler Hintergrundprozesse}
    \label{tab:fehler_hintergrundprozesse}
\end{longtable}

\subsubsection{Cronjob Fehler}
\label{ssub:cronjob_fehler}
Ältere Webprojekte benötigen für Aufräumarbeiten einen oder mehrere Cronjobs. Fehler in diesen Cronjobs haben nicht in allen Fällen einen direkten Einfluss auf das Funktionieren der Webseite. Jedoch kann es zu Problemen kommen falls ein solcher Job längere Zeit ausfällt.

\subsubsection{Worker Fehler}
\label{ssub:worker_fehler}
Background Worker werden eingesetzt um rechenintensive Arbeiten aus der Webapplikation auszulagern. Falls dieser Worker nicht mehr läuft, oder ein Task nicht endet kann dies dazu führen, dass sich die Taskqueue füllt und neue Tasks nicht mehr abgearbeitet werden.

\subsection{Applikation}
\label{sub:fehler_applikation}

\begin{longtable}{l>{\raggedright}p{8cm} r r}
    \toprule \textbf{Nr.} & \textbf{Bezeichnung} & \textbf{Impact} & \textbf{Frequenz} \\
    \newfnumber{Deprecated Librarys}{deprecated_librarys}{1}{2}
    \newfnumber{Unittest Fehler}{unittest_fehler}{3}{3}
    \newfnumber{Fehler im Produktivsystem}{fehler_im_produktivsystem}{2}{2}
    \newfnumber{Missverhalten}{missverhalten}{2}{2}
    \newfnumber{Debug Modus}{debug_modus}{2}{2}
    \newfnumber{Abhängigkeiten mit Sicherheitslücken}{abhängigkeiten_mit_sicherheitslücken}{3}{2}
    \newfnumber{404 Handling nicht falsch}{404_handling_nicht_falsch}{1}{1}
    \newfnumber{Datenbank Queries laufen langsam}{datenbank_queries_laufen_langsam}{1}{1}
    \newfnumber{Applikation läuft langsam}{applikation_läuft_langsam}{1}{1}
    \bottomrule
    \caption[Mögliche Fehler Applikation]{Mögliche Fehler Applikation}
    \label{tab:fehler_applikation}
\end{longtable}

\subsubsection{Deprecated Librarys}
\label{ssub:deprecated_librarys}
In Webprojekten werden häufig Librarys von Drittherstellern verwendet. Falls in einer Library eine Funktion entfernt wird, wird dies meist einige Versionen vor dem eigentlichen Entfernen durch eine ``Deprecation Warning'' angekündigt. Diese Warnungen sollten durch den Programmierer bemerkt werden und er sollte wo möglich auf das Verwenden dieser Funktionen verzichten.

\subsubsection{Unittest Fehler}
\label{ssub:unittest_fehler}

\subsubsection{Fehler im Produktivsystem}
\label{ssub:fehler_im_produktivsystem}
Ein Fehler beim bearbeiten eines Requests wird dem Benutzer mit einer Meldung und dem Statuscode 500 mitgeteilt. Da nicht alle Benutzer solche Fehler melden wird zusätzlich ein Fehlerprotokoll angelegt.

\subsubsection{Missverhalten}
\label{ssub:missverhalten}
Als Missverhalten wird jegliches Verhalten einer Webseite angesehen welches nicht dem vom Entwickler vorgesehenen Verhalten entspricht. Dabei muss es sich nicht zwingend um einen Fehler im Programmcode handeln.

\subsubsection{Bedug Modus}
\label{ssub:bedug_modus}
Die meisten Webframeworks besitzen einen Debug Modus. Dieser Debug Modus ist für Testsysteme und Entwicklungssysteme vorgesehen. Da Applikationen im Debug Modus nicht optimal laufen und auch Memory Leaks auftreten können sollte dieser nicht in Produktiv eingesetzten Systemen verwendet werden.

\subsubsection{Abhängigkeiten mit Sicherheitslücken}
\label{ssub:abhängigkeiten_mit_sicherheitslücken}
Da Webapplikationen meistens viele Abhängigkeiten in Form von Programmbibliotheken haben, sind sie von Sicherheitslücken der verwendeten Bibliotheken betroffen. Um zu verhindern, dass eine Webapplikation von einer Sicherheitslücke betroffen ist, muss die verursachende Abhängigkeit durch eine neuere Version welche diese Lücke schliesst ersetzt werden.


\section{Frontend}
\label{sec:frontend}

\begin{longtable}{l>{\raggedright}p{8cm} r r}
    \toprule \textbf{Nr.} & \textbf{Bezeichnung} & \textbf{Impact} & \textbf{Frequenz} \\
    \newfnumber{Javascript Fehler}{javascript_fehler}{2}{3}
    \newfnumber{CSS Fehler}{css_fehler}{3}{1}
    \newfnumber{Seite lädt zu langsam}{site_lädt_zu_langsam}{2}{4}
    \newfnumber{Browser spezifische Probleme}{browser_spezifische_probleme}{2}{2}
    \newfnumber{Assets fehlen}{assets_fehlen}{2}{3}
    \newfnumber{Externe Abhängigkeiten nicht verfügbar}{externe_abhängigkeiten}{3}{3}
    \newfnumber{Seite funktioniert nicht auf mobilen Geräten}{seite_funktioniert_nicht_auf_mobilen_geräten}{3}{1}
    \bottomrule
    \caption[Mögliche Fehler Frontend]{Mögliche Fehler Frontend}
    \label{tab:fehler_frontend}
\end{longtable}

\subsubsection{Javascript Fehler}
\label{ssub:javascript_fehler}

\subsubsection{CSS Fehler}
\label{ssub:css_fehler}

\subsubsection{Seite lädt zu langsam}
\label{ssub:seite_lädt_zu_langsam}

\subsubsection{Browser spezifische Probleme}
\label{ssub:browser_spezifische_probleme}

\subsubsection{Assets fehlen}
\label{ssub:assets_fehlen}
Für die Darstellung einer Webseite werden meistens zusätzlich zur eigentlichen HTML-Datei noch Bilder, Stylesheets und Javascripts verwendet. Diese zusätzlichen Dateien, auch Assets genannt, werden vom Browser geladen. Sind im Quelltext einer Webseite Assets vermerkt welche nicht verfügbar sind, zögert dies die Darstellung der Webseite unnötig hinaus.

\subsubsection{Externe Abhängigkeiten nicht verfügbar}
\label{ssub:externe_abhängigkeiten_nicht_verfügbar}

\subsubsection{Seite funktioniert nicht auf mobilen Geräten}
\label{ssub:seite_funktioniert_nicht_auf_mobilen_geräten}


\section{Inhalt}
\label{sec:inhalt}

\begin{longtable}{l>{\raggedright}p{8cm} r r}
    \toprule \textbf{Nr.} & \textbf{Bezeichnung} & \textbf{Impact} & \textbf{Frequenz} \\
    \newfnumber{Seite enthält tote Links}{seite_enthält_tote_links}{1}{3}
    \newfnumber{Rechtschreibefehler}{rechtschreibefehler}{1}{2}
    \newfnumber{Falsch aufbereitete Bilder}{falsch_aufbereitete_bilder}{1}{2}
    \newfnumber{Design verletzt}{design_verletzt}{1}{2}
    \newfnumber{Fehlmanipulation durch den Kunden}{fehlmanipulation_durch_den_kunden}{2}{2}
    \bottomrule
    \caption[Mögliche Fehler Inhalt]{Mögliche Fehler Inhalt}
    \label{tab:fehler_inhalt}
\end{longtable}

\subsubsection{Seite enthält tote Links}
\label{ssub:seite_enthält_tote_links}
Da URL's im Internet einem ständigen Wandel unterstellt sind, kann es vor kommen, dass ein Link plötzlich ins Leere zeigt. Solche Links werden als tote Links bezeichnet und sind normalerweise nicht erwünscht, da sie die Benutzer einer Webseite auf eine Fehlermeldung weiterleiten.

\subsubsection{Rechtschreibefehler}
\label{ssub:rechtschreibefehler}

\subsubsection{Falsch aufbereitete Bilder}
\label{ssub:falsch_aufbereitete_bilder}

\subsubsection{Design verletzt}
\label{ssub:design_verletzt}

\subsubsection{Fehlmanipulation durch den Kunden}
\label{ssub:fehlmanipulation_durch_den_kunden}


\section{SEO and Sharing}
\label{sec:seo_and_sharing}

\begin{longtable}{l>{\raggedright}p{8cm} r r}
    \toprule \textbf{Nr.} & \textbf{Bezeichnung} & \textbf{Impact} & \textbf{Frequenz} \\
    \newfnumber{OG Tags fehlerhaft oder nicht vorhanden}{og_tags_fehlerhaft_oder_nicht_vorhanden}{2}{1}
    \newfnumber{``Meta Description'' und Seitentitel fehlerhaft oder nicht vorhanden}{meta_description_und_seitentitel}{2}{2}
    \newfnumber{Deeplinks funktionieren nicht}{deeplinks_funktionieren_nicht}{2}{1}
    \bottomrule
    \caption[Mögliche Fehler SEO und Sharing]{Mögliche Fehler SEO und Sharing}
    \label{tab:fehler_seo_sharing}
\end{longtable}

\subsubsection{OG Tags fehlerhaft oder nicht vorhanden}
\label{ssub:og_tags_fehlerhaft_oder_nicht_vorhanden}
Die ursprünglich von Facebook eingeführten OG Tags erlauben es einer Webseite zusätzliche Informationen anzuhängen. Diese Informationen werden von verschiedenen sozialen Netzwerken verwendet um Links mit einer Beschreibung und einem Bild anzureichern. Fehlen diese Tags oder sind sie fehlerhaft, werden die falschen oder gar keine zusätzliche Informationen angezeigt was zu einer niedrigeren Klick-Rate führen kann.

\subsubsection{``Meta Description'' und Seitentitel fehlerhaft oder nicht vorhanden}
\label{ssub:_meta_description_und_seitentitel_fehlerhaft_oder_nicht_vorhanden}
Suchmaschinen greifen für die Anzeige der Resultate auf den Seitentitel und die ``Meta Description'' zurück. Fehlen diese Angaben versuchen Suchmaschinen den Titel und die Beschreibung aus einer Webseite zu extrahieren. Diese Resultate sind jedoch meist weniger gut als von Hand erfasste Titel und Beschreibungen. Dies führt dazu, dass Benutzer weniger auf diese Suchergebnisse klicken.

\subsubsection{Deeplinks funktionieren nicht}
\label{ssub:deeplinks_funktionieren_nicht}

\section{Programmcode}
\label{sec:programmcode}

\begin{longtable}{l>{\raggedright}p{8cm} r r}
    \toprule \textbf{Nr.} & \textbf{Bezeichnung} & \textbf{Impact} & \textbf{Frequenz} \\
    \newfnumber{Code Conventions nicht eingehalten}{code_conventions}{1}{4}
    \newfnumber{Programmcode enthält Syntaxfehler}{syntaxfehler}{3}{3}
    \bottomrule
    \caption[Mögliche Fehler Programmcode]{Mögliche Fehler Programmcode}
    \label{tab:fehler_programmcode}
\end{longtable}

\subsubsection{Code Conventions nicht eingehalten}
\label{ssub:code_conventions_nicht_eingehalten}
Um häufiges Umformatieren von Programmcode zu vermeiden sollten sich alle Programmierer innerhalb eines Projektes an die jeweils gültigen Code Konventionen halten. Diese bestehen meist aus einem Style-Guide und Namenskonventionen.

\subsubsection{Programmcode enthält Syntaxfehler}
\label{ssub:programmcode_enthält_syntaxfehler}
In Programmiersprachen welche durch einen Interpreter ausgeführt werden, werden nicht zwingend alle Syntaxfehler und Tippfehler während der Entwicklung gefunden. Solche Fehler sollten jedoch gefunden werden bevor ein Programm in einer Produktivumgebung eingesetzt wird.
