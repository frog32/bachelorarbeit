%!TEX root = ../index.tex

\section{Nomenklatur}
\label{sec:nomenklatur}

Um die Lesbarkeit zu erhöhen, werden in dieser Arbeit folgende Begriffe für bestimmte Personen oder Rollen verwendet:

{\bf Kunde} steht für einen Kunden von allink GmbH. Dies steht meistens für die Person welche auf Kundenseite für das Webprojekt zuständig ist.

{\bf Benutzer} steht für einen Benutzer der Webseite. Diese Person ist in den meisten Fällen nicht genauer bekannt und ein potentieller Kunde des Kunden von allink GmbH.

\section{Ausgangslage}
\label{sec:ausgangslage}

Webprojekte geniessen gewöhnlich während ihrer Entstehungsphase eine hohe Aufmerksamkeit. Der zugrundeliegende Programmcode wird meist mit Unittests geprüft um sicherzustellen, dass im Falle von zukünftigen Änderungen keine Regressionen nach sich ziehen. Webseiten werden zum Zeitpunkt der Abnahme durch den Kunden meist einer Gründlichen Prüfung unterzogen, sowohl durch den Programmierter wie auch durch den Kunden. Nach der Abnahme und nachdem das Webprojekt online verfügbar ist, werden während den ersten Wochen die Zugriffsstatistiken und die Suchmaschienen-Platzierungen analysiert, falls diese Analyse grössere Mängel aufdeckt werden diese normalerweise behoben.

Nach dieser ersten Phase werden Inhalte einer Webseite meist von Personen ohne technisches Wissen unterhalten. Dieses Fehlen von technischem Wissen und die Tatsache, dass sich das Internet um die Webseite weiterentwickelt ist häufig der Grund Probleme welche sich ergeben. Solche Probleme können Kleinigkeiten sein, wie die längere Ladezeit einer Webseite verursacht durch ein nicht fachgerecht komprimiertes Bild, oder dazu führen, dass eine Webseite nicht mehr verfügbar ist.

\section{Problemstellung}
\label{sec:problemstellung}

Zu beginn der Phase in der ein Kunde seine Inhalte in eine Webseite abfüllt, findet jeweils eine Schulung statt. Dabei wird, denjenigen welche die Webseite mit Inhalten füllen, erklärt wie Inhalte eingepflegt werden müssen. In dieser Phase wird versucht alle Unklarheiten welche den Umgang mit dem \ac{cms} betreffen zu klären. Jedoch wird bei vielen Webseiten der Inhalt regelmässig verändert oder aktualisiert. Da die Person welche für den Inhalt der Webseite zuständig ist, meist nicht ausschliesslich für die Webseite zuständig ist, sind einfache Vorgänge wie das ersetzen eines Textes nach einiger Zeit nicht mehr selbstverständlich. Dadurch werden Inhalte teils nicht wie ursprünglich vorgesehen eingefüllt. Im optimalen Fall hat dies keinen direkten Einfluss auf das Funktionieren der Webseite und steht einem Benutzen der Webseite nicht im Wege. Es kann aber vorkommen, dass dadurch Teile der Webseite nicht mehr richtig funktionieren. Dies kann sich durch eine falsche Darstellung, verminderter Leistung oder im Extremfall durch eine nicht mehr benutzbare Webseite zeigen.

Im besten Fall fällt es dem Kunden auf, dass die Webseite nicht richtig funktioniert und das Problem kann behoben werden. Falls das Problem für den Kunden nicht unmittelbar sichtbar ist, kann es vorkommen, dass die Webseite zwar den Anschein macht, richtig zu funktionieren jedoch wichtige Funktionen gestört werden. Dies kann zum Beispiel bedeuten, dass die Webseite nicht mehr von Suchmaschinen indiziert werden kann.

\section{Ziel der Arbeit}
\label{sec:ziel_der_arbeit}
