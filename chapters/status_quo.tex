%!TEX root = ../index.tex

Die IT Abteilung in der allink wurde 2010 komplett neu gebildet. Zu diesem Zeitpunkt wurden ebenfalls grundlegende Technologie-Entscheide neu gefällt.

\section{Eingesetzte Technologie}
\label{sec:eingesetzte_technologie}
Der grösste Teil aller Webprojekte in der allink werden mit Hilfe des Django Frameworks realisiert. Durch diese Framework-Wahl ergibt sich Python als präferierte Programmiersprache für alle Server-seitigen Applikationen.

\section{Produktiv Systeme}
\label{sec:produktiv_systeme}
Die allink setzt für alle produktiven Web Systeme auf Managed Hosting von Nine Internet Solutions AG \footnote{http://nine.ch}. Von Nine wird die Verfügbarkeit der Serversysteme sichergestellt und Unterhaltsarbeit geleistet. Auch Backups der produktiven Systeme werden von Nine erstellt und archiviert.

\begin{table}[h!]
  \centering
  \begin{tabular}{ll}
  \toprule
    Softwarepaket & Version\\
  \midrule
    Apache & 2.2\\
  \hline
    Python & 2.7\\
  \hline
    libapache2-mod-wsgi & 3.3\\
  \hline
    MySQL Server & 5.5\\
  \hline
    Redis & 2.2\\
  \hline
    RabbitMQ & 2.7\\
  \bottomrule
  \end{tabular}
  \caption{Eingesetzte Softwareversionen}
  \label{tab:eingesetzte_softwareversionen}
\end{table}

\section{Eingesetzte Mittel für die Qualitätssicherung}
\label{sec:eingesetzte_mittel_für_die_qualitätssicherung}
Qualitätssicherung ist in der allink schon immer ein Thema. In der Zeit seit dem die allink Django einsetzt wurden einige Bemühungen getätigt um die Qualität zu gewährleisten. Die Tabelle~\ref{tab:qm_eingesetzte_mittel} zeigt eine Übersicht der Mittel welche standardmässig bei jedem Webprojekt zum Einsatz kommen. Diese sind meist in das allink-project, ein Django Projekt welches die Grundlage für alle in der allink entwickelten Webapplikationen ist, integriert.

\makeatletter
\newcounter{bcounter}
\newcounter{bnumber} \setcounter{bnumber}{0}
\renewcommand\thebnumber{B\arabic{bnumber}}
\newcommand{\newbnumber}[3]%
{%
\midrule%
\refstepcounter{bnumber}\label{b:#2}%
\expandafter\xdef\csname b#2\endcsname {#1}%
\thebnumber & #1 & #3 \\
}
\makeatother


\begin{longtable}{l>{\raggedright}p{3cm}p{10cm}}
    \toprule \textbf{Nr.} & \textbf{System} & \textbf{Beschreibung} \\
    \newbnumber{unittests}{Unittests}{Business Logik wird wo vorhanden mittels Unittests getestet}
    \newbnumber{Precommit Hook}{precommithook}{Quellcode wird vor dem einchecken statisch ausgewertet und auf Konformität mit dem Syleguide geprüft.}
    \newbnumber{Sentry}{sentry}{Fehlermeldungen sämtlicher Webprojekte werden zentral mit Sentry aggregiert.}
    \newbnumber{GoLive-Checkliste}{golivecheckliste}{Für den Go-Live einer Webseite besteht ein ausführliches Testprotokoll.}
    \newbnumber{Deployment Prozess}{deploymentprozess}{Der Deployment-Prozess wurde so implementiert, dass er einige Probleme selber erkennen kann.}
    \bottomrule
    \caption[Eingesetze Mittel]{Eingesetzte Mittel}
    \label{tab:qm_eingesetzte_mittel}
\end{longtable}
