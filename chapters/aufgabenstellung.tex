%!TEX root = ../index.tex
\section{Ausgangslage}
\label{sec:anhang_ausgangslage}
Allink ist eine Agentur welche unter anderem Webseiten und Webapplikationen für ihre Kunden erstellt. Diese Projekte sind in der Regel zwischen einem Monat und mehreren Jahren aktiv und sollen auch mit verändertem Inhalt und auf verschiedenen Browsern korrekt funktionieren. Es existiert bereits ein Qualitätsmanagement, welches mittels \glspl{unittest} und eines Abschlusstests die Qualität zum Zeitpunkt der Abnahme eines Projektes sicherstellt.

\section{Ziel der Arbeit}
\label{sec:anhang_ziel_der_arbeit}
Es soll ein allgemeines Konzept erstellt werden, welches die Qualitätssicherung von Webprojekten nach der Abnahme durch den Kunden ermöglicht. Fehler sollen auch nach dem Ende der Programmierphase eines Projektes erkannt werden.

\section{Aufgabenstellung}
\label{sec:anhang_aufgabenstellung}
\begin{itemize}
  \item Erarbeiten eines Konzeptes zum Testen von Webprojekten, sowohl für automatisiert testbare, als auch nicht automatisiert testbare Fehlerszenarios. Dies beinhaltet die Analyse möglicher Fehlerszenarios (z.B. für verschiede Typen von Webseiten / - Applikationen) und deren Katalogisierung.
  \item Anhand des Konzeptes sollen Anforderungen an eine Software, welche die automatisiert testbaren Fehlerszenarios abdecken kann, aufgestellt werden.
  \item Aufgrund der Anforderungsanalyse soll eine Marktanalyse durchgeführt werden. Ziel ist bestgeeignete Softwarekomponenten zu finden.
  \item Proof of Concept: Zum Beweis der Machbarkeit werden die evaluierten Softwarekomponenten und wo nötig Eigenentwicklungen in der allink eingeführt.
  \item Es soll ein technischer Bericht erstellt werden.
\end{itemize}

\section{Erwartetes Resultat}
\label{sec:anhang_erwartetes_resultat}
\begin{itemize}
  \item Katalog der häufigsten Fehlerszenarios mit Eintrittswahrscheinlichkeiten und Tragweiten
  \item Kategorisierung der Fehlerszenarios in automatisiert und manuell testbar
  \item Anforderungsanalyse
  \item Marktanalyse
  \item Proof of Concept: Anwendung des Konzepts auf die Allink
  \item Technischer Bericht
\end{itemize}

\section{Geplante Termine}
\label{sec:anhang_geplante_termine}
\begin{itemize}
  \item Kick-Off
  \item Design Review
  \item Schlusspräsentation ZHAW
\end{itemize}
