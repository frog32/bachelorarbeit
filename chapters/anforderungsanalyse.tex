%!TEX root = ../index.tex

\section{Systemkontext}
\label{sec:systemkontext}
Der Systemkontext für Programme welche die Qualität von Webprojekten überprüfen und sicherstellen ist für alle Programme der selbe. In diesem Systemkontext sind bereits mehrere Programme und Systeme vorhanden. Die neu einzuführenden Systeme sollen wo möglich mit den bereits vorhandenen zusammenarbeiten.

\subsection{allink.planer}
\label{sub:allink_planer}
Der allink.planer ist ein Programm welches in der allink entwickelt wurde. Im allink Planer sind zur Zeit diverse Funktionen implementiert. Die Funktionalitäten im allink.planer welche den für diese Arbeit relevant sind werden im folgenden kurz erklärt. Weiterhin sind im allink.planer vor allem Funktionalitäten für die Auswertung von Projekten implementiert sowie auch ein Rechnungs und Offertensystem.

\subsubsection{Hosting Tracking}
\label{ssub:hosting_tracking}
Um die jährlichen Hostingrechnungen zu automatisieren, wurde in den allink.planer ein Programm eingebaut welches sämtliche Produktivserver der allink Täglich überprüft und alle darauf existierenden Webprojekte sammelt und ein Verzeichnis dieser aktuell hält. In diesem System sind alle für diese Arbeit relevanten Webprojekte verzeichnet. Wo möglich sollen neue Systeme mit diesem Verzeichnis arbeiten, damit neue Webprojekte nicht von Hand erfasst werden müssen.

\subsection{Produktivsysteme}
\label{sub:produktivsysteme}


\section{Stakeholder}
\label{sec:stakeholder}

\subsection{Programmierer}
\label{sub:programmierer}


\subsection{Geschäftsleitung}
\label{sub:geschäftsleitung}
Die Projektleitung besteht in der allink hauptsächlich aus den vier Teilhabern. Dadurch sind die Anforderungen der Projektleitung deckend mit denen der Geschäftsleitung, da es sich hier um die selben Personen handelt. Die Geschäftsleitung erwartet, dass die Kosten welche mit den neuen Systemen entstehen in einem angemessenen Rahmen bleiben.

\subsection{Kunde}
\label{sub:kunde}
Kunden der allink erwarten, dass ihre Webprojekte möglichst ohne Fehler laufen und dauernd verfügbar sind. Da viele Kunden kein oder nur wenig Wissen im Bereich der Informatik haben, können sie teils nur schwer nachvollziehen warum ein System auch nach dem die Programmierarbeiten abgeschlossen sind weiter betreut werden muss. Sie erwarten, dass trotz ändernden Bedingungen durch neue Browser ihre Webseite auf den häufigsten Browsern richtig dargestellt wird und falls ein Problem auftritt dieses möglichst schnell behoben wird.

\begin{itemize}
  \item Anforderungen an ein Tool zum automatisierten Testen
  \item Anforderungen an einen Manuellen Testablauf
\end{itemize}
