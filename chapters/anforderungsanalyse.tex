%!TEX root = ../index.tex

\newcounter{anumber} \setcounter{anumber}{0}
\renewcommand\theanumber{A\arabic{anumber}}
\newcommand{\newanumber}[3]%
{%
\midrule%
\refstepcounter{anumber}\label{a:#1}%
\theanumber & #2 & #3 \\
}


\section{Systemkontext}
\label{sec:systemkontext}
Der Systemkontext für Programme welche die Qualität von Webprojekten überprüfen und sicherstellen ist für alle Programme der selbe. In diesem Systemkontext sind bereits mehrere Programme und Systeme vorhanden. Die neu einzuführenden Systeme sollen wo möglich mit den bereits vorhandenen zusammenarbeiten.

\subsection{allink.planer}
\label{sub:allink_planer}
Der allink.planer ist ein Programm welches in der allink entwickelt wurde. Im allink Planer sind zur Zeit diverse Funktionen implementiert. Die Funktionalitäten im allink.planer welche den für diese Arbeit relevant sind werden im folgenden kurz erklärt. Weiterhin sind im allink.planer vor allem Funktionalitäten für die Auswertung von Projekten implementiert sowie auch ein Rechnungs und Offertensystem.

\subsubsection{Hosting Tracking}
\label{ssub:hosting_tracking}
Um die jährlichen Hostingrechnungen zu automatisieren, wurde in den allink.planer ein Programm eingebaut welches sämtliche Produktivserver der allink Täglich überprüft und alle darauf existierenden Webprojekte sammelt und ein Verzeichnis dieser aktuell hält. In diesem System sind alle für diese Arbeit relevanten Webprojekte verzeichnet. Wo möglich sollen neue Systeme mit diesem Verzeichnis arbeiten, damit neue Webprojekte nicht von Hand erfasst werden müssen.

\subsection{Produktivsysteme}
\label{sub:produktivsysteme}
Mit Produktivsysteme sind die Server gemeint auf welchen die Webapplikationen welche in der allink entwickelt werden gehostet sind. Im Normalfall handelt es sich hier um virtuelle Server welche von der Firma Nine Internet Solutions AG betrieben und unterhalten werden. Dadurch ist die Verantwortung für den Betrieb der Server nicht bei der allink und die Software Entwickler können sich auf die Applikation konzentrieren. Um den Betrieb der Server zu kontrollieren setzt Nine Internet Solutions eine monitoring Software ein. Diese Software ist jedoch ausserhalb des Systemkontext und sollte daher keinen Einfluss diese Arbeit haben.

\section{Stakeholder}
\label{sec:stakeholder}


\subsection{Programmierer}
\label{sub:programmierer}
Die Programmierer in der allink setzen meist mehrere Projekte pro Person pro Monat um. Vor allem bei kleineren Projekten, kann es vorkommen, dass nach wenigen Tagen das Projekt abgeschlossen ist. Da viele Webseiten nach einiger Zeit kleine Aktualisierungen benötigen, ist es wichtig, dass alle Projekte möglichst die selbe Projektstruktur haben. Die Programmierer erwarten daher, dass die neuen Systeme zur Erhaltung der Qualität gut in das bestehende Setup integriert werden können.

\subsection{Projektleitung}
\label{sub:projektleitung}
Die Projektleitung besteht in der allink hauptsächlich aus den vier Teilhabern. Diese erwarten, dass sie durch die neuen Systeme keinen Mehraufwand für sich selber erhalten. Da sich erhöhte Qualität bei Kunden schlecht als Mehrwert verkaufen lässt, erwartet die Projektleitung, dass sich der Mehraufwand welche für jedes Projekt anfällt durch die Ersparnisse durch verfrühtes Erkennen lohnt. Es sollen jedoch in Zukunft Fehler und Probleme wo möglich erkannt werden bevor der Kunde diese entdeckt.

\subsection{Geschäftsleitung}
\label{sub:geschäftsleitung}
Die Geschäftsleitung erwartet, dass die Kosten welche mit den neuen Systemen entstehen in einem angemessenen Rahmen bleiben.

\subsection{Kunde}
\label{sub:kunde}
Kunden der allink erwarten, dass ihre Webprojekte möglichst ohne Fehler laufen und dauernd verfügbar sind. Da viele Kunden kein oder nur wenig Wissen im Bereich der Informatik haben, können sie teils nur schwer nachvollziehen warum ein System auch nach dem die Programmierarbeiten abgeschlossen sind weiter betreut werden muss. Sie erwarten, dass trotz ändernden Bedingungen durch neue Browser ihre Webseite auf den häufigsten Browsern richtig dargestellt wird und falls ein Problem auftritt dieses möglichst schnell behoben wird.

\subsection{Hostingprovider}
\label{sub:hosting_provider}
Die Nine Internet Solutions AG ist Hostingprovider für alle Produktivsysteme der allink. Nine will weiterhin eine klare Trennung der Zuständigkeit wie dies bis anhin der Fall war. Es sollen zudem keine neuen Aufgaben für Nine entstehen.

\section{Anforderungen}
\label{sec:anforderungen}

Für alle Anforderungen wird von 100 Webapplikationen verteilt auf 5 Server ausgegangen.

\begin{longtable}{l>{\raggedright}p{4cm} p{8cm}}
    \toprule \textbf{Nr.} & \textbf{Quelle} & \textbf{Beschreibung} \\
    \newanumber{einfach_implementierbar}{Programmierer}{Sämtliche zusätzliche Systeme sollen einfach in den allink Programmierprozess integrierbar sein.}
    \newanumber{kosten}{Geschäftsleitung}{Für die bestehenden Produktivsysteme sollen die Kosten für externe Systeme nicht CHF 6000 pro Jahr übersteigen.}
    \newanumber{aufwand}{Projektleitung}{Es sollen auf Projektebene kein Mehraufwand entstehen.}
    \newanumber{sicherheit}{Programmierer}{Alle externen Tools welche Benutzerdaten benötigen sollen über eine sichere Verbindung verfügbar sein.}
    \newanumber{hosting}{Hostingprovider}{Es sollen keine neuen Zuständigkeiten und Aufgaben für den Hostingprovider entstehen.}
    \bottomrule
    \caption[Anforderungen]{Anforderungen}
    \label{tab:anforderungen}
\end{longtable}

Zusätzlich zu diesen Anforderungen, sollen alle eingesetzten Tools zusammen mit den bestehenden Mitteln und den übrigen Neuerungen den gesamten Fehlerkatalog aus Kapitel~\ref{cha:fehlerkatalog} abdecken. 
